\sloppy


% Добавляем гипертекстовое оглавление в PDF и метаданные
\usepackage[
bookmarks=true, colorlinks=true, unicode=true,
urlcolor=black,linkcolor=black, anchorcolor=black,
citecolor=black, menucolor=black, filecolor=black,
pdfstartpage=1,
    pdftex,
            pdfauthor={Зотолов Владислав Владимирович},
            pdftitle={ },
            pdfsubject={БрГТУ МС-7 \the\year},
            pdfkeywords={},            
]{hyperref}
%% возможные значения по умолнанию
% Приложение:           pdfproducer={pdfTeX-1.40.25},
% Производитель PDF:    pdfcreator={LaTeX with hyperref}
%%


% Изменение начертания шрифта --- после чего выглядит таймсоподобно.
% apt-get install scalable-cyrfonts-tex

\IfFileExists{cyrtimes.sty}
    {
        \usepackage{cyrtimespatched}
    }
    {
        % А если Times нету, то будет CM...
    }


\usepackage{graphicx}   % Пакет для включения рисунков
\DeclareGraphicsExtensions{.jpg,.pdf,.png}
% \graphicspath{{Source/img/}}    % расположение рисунков % было перенесено в main
\usepackage{float} % Для опции фигур [H]


% С такими оно полями оно работает по-умолчанию:
%\RequirePackage[left=20mm,right=10mm,top=20mm,bottom=20mm,headsep=0pt]{geometry}
% Если вас тошнит от поля в 10мм --- увеличивайте до 20-ти, ну и про переплёт не забывайте:
\geometry{right=20mm}
\geometry{left=30mm}
\geometry{headsep=5mm} % чтоб код зачетки не налазил на главы


% Настройки стиля ГОСТ 7-32
% Для начала определяем, хотим мы или нет, чтобы рисунки и таблицы нумеровались в пределах раздела, или нам нужна сквозная нумерация.
\EqInChapter % формулы будут нумероваться в пределах раздела
\TableInChapter % таблицы будут нумероваться в пределах раздела
\PicInChapter % рисунки будут нумероваться в пределах раздела


% Произвольная нумерация списков.
\usepackage{enumerate}

\setcounter{tocdepth}{1} %Подробность оглавления
%5 это chapter, section, subsection, subsubsection и paragraph
%4 это chapter, section, subsection и subsubsection
%3 это chapter, section, и subsection
%2 это chapter и section
%1 это chapter.
%0 ничего


\usepackage{subfigure} %https://latex-tutorial.com/subfigure-latex/
%%
%Необходима для правильности отображения ссылок на на subfigure при нумерации figure по главам
% https://tex.stackexchange.com/questions/115712/figure-numbering-with-section-number-problem-with-subfigure
\makeatletter 
\renewcommand\p@subfigure{\thefigure}
\makeatother
%%


%% настройка листинга
\usepackage{listings}

\lstdefinestyle{mystyle}{
    language=[Sharp]C, % язык программирования
    frame=none, % без рамки
    xleftmargin=1mm, % отступ слева
    xrightmargin=1mm, % оступ справа
    breaklines=true, % переносить строки
    extendedchars=false,
    keepspaces=true,
    basicstyle=\normalsize	
}
% Возможные парамеметры basicstyle , настраиваюшие размер шрифта
%\tiny	
%\scriptsize	
%\footnotesize	
%\small	
%\normalsize	
%\large
%\Large	
%\LARGE	
%\huge	
%\Huge	
\lstset{style=mystyle}
%%

% настройка отображения источников \nocite означает что они будут добавленны даже если в тексте не использовались ссылки.
% если закомментировать, то отображаться будут только те источники на которых ссылались
\nocite{*}
%%


% Расскометируйте строчку ниже если хотите добавить фон в весь файл
% Тогда каждой странице исходного файла будет соответсвовать страница файла с фоном
% Размер файла с фонами, в страницах, должен быть >= размеру  исходного файла
% Данный макрос взят отсюда: https://habr.com/ru/articles/163315/
%\newcommand{\overlaypass}

\usepackage{eso-pic}

\ifdefined\overlaypass
\newcounter{overlaypage}
\setcounter{overlaypage}{1}
\AddToShipoutPicture{\AtPageLowerLeft{\includegraphics[page=\arabic{overlaypage}]
{../Source/background/background.pdf}\stepcounter{overlaypage}}}
\fi
%%


% установка стиля страниц
% изначально определлены в файле G2-105.sty : headings и myheadings
\pagestyle{headingsWithCode}
%%